\documentclass[11pt,leqno]{article}
\usepackage[spanish,activeacute]{babel}
\usepackage[utf8]{inputenc}
\usepackage{amsfonts}
\usepackage{enumerate}
\usepackage{listings}
\usepackage{amsthm}
\usepackage[hidelinks]{hyperref}
\usepackage{xcolor}

\definecolor{color0}{RGB}{147, 147, 147} %<--- I've changed this to make it more visible
\definecolor{color1}{RGB}{186, 033, 033}
\definecolor{color2}{RGB}{000, 128, 000}
\definecolor{color3}{RGB}{064, 128, 128}
\definecolor{color4}{RGB}{170, 034, 255}


\lstdefinelanguage{clips}{
  morekeywords = {deffunction, deftemplate, defrule, deffacts, run,
    clear, reset, facts, agenda, nil, initial-fact, assert, retract,
    watch, ppdefrule, unwatch, crlf},
  sensitive        = true,
  morecomment      = [l]{;},
  morestring       = [b]",
  basicstyle       = \ttfamily\small,
  numbers          = left,
  numberstyle      = \tiny,
  showstringspaces = false,
  keywordstyle=\color{blue}\bfseries,
  stringstyle=\color{cyan}\ttfamily,
  frame=L,
  xleftmargin=\parindent,
  commentstyle=\color{teal}\ttfamily,
}

% egreg's modulo macro (see http://tex.stackexchange.com/a/34449/21891)
\def\truncdiv#1#2{((#1-(#2-1)/2)/#2)}
\def\moduloop#1#2{(#1-\truncdiv{#1}{#2}*#2)}
\def\modulo#1#2{\number\numexpr\moduloop{#1}{#2}\relax}


\makeatletter

% a TeX counter to keep track of the nesting level
\newcount\netParensCount@clisp

% Modify how ( and ) get typeset depending on the value of the counter
% (Based on Ulrike Fischer's approach to modifying characters in listings;
% see http://tex.stackexchange.com/a/231927/21891)
\lst@CCPutMacro
\lst@ProcessOther{`(}{{%
  \ifnum\lst@mode=\lst@Pmode\relax%
    \rainbow@clisp{(}%
    \global\advance\netParensCount@clisp by \@ne%
  \else
    (%
  \fi
}}%
\lst@ProcessOther{`)}{{%
  \ifnum\lst@mode=\lst@Pmode\relax%
    \global\advance\netParensCount@clisp by \m@ne%
    \rainbow@clisp{)}%
  \else
    )%
  \fi
}}%
\@empty\z@\@empty

% Color its argument based on the value of the \netParensCount@clisp counter
% (modulo 5)
\newcommand\rainbow@clisp[1]{%
  \ifcase\modulo\netParensCount@clisp 5\relax%
    \textcolor{color0}{#1}%
  \or
    \textcolor{color1}{#1}%
  \or
    \textcolor{color2}{#1}%
  \or
    \textcolor{color3}{#1}%
  \else
    \textcolor{color4}{#1}%
  \fi
}

% Alternatively, you could simplify the definition of \rainbow@clisp to...
% \newcommand\rainbow@clisp[1]{%
%   \textcolor{color\modulo\netParensCount@clisp 5}{#1}%
% }
% ... but this assumes that the colours have names of the form color<i>,
% where <i> is a positive integer

% reset the counter at the beginning of each listing
% (just in case there were unmatched parentheses in a previous listing)
\lst@AddToHook{PreInit}{%
  \global\netParensCount@clisp 0\relax%
}

\makeatother


\lstnewenvironment{clips-code}
  {\lstset{language=clips}}
  {}

% Título y autor
\title{Práctica final\\
		Ingeniería del conocimiento\\
		Desarrollo de un sistema experto}
		
\author{Jacinto Carrasco Castillo 	\\
		N.I.F. 32056356-Z			\\ 
		\href{jacintocc@correo.ugr.es}{jacintocc@correo.ugr.es}}
		
% Definición de estilo de definición
\newtheoremstyle{definition_wo_parentheses}
  {\topsep}% measure of space to leave above the theorem. E.g.: 3pt
  {\topsep}% measure of space to leave below the theorem. E.g.: 3pt
  {}% name of font to use in the body of the theorem
  {0pt}% measure of space to indent
  {\bfseries}% name of head font
  {.}% punctuation between head and body
  { }% space after theorem head; " " = normal interword space
  {\thmname{#1}\thmnumber{ #2.}\thmnote{ #3}}
  
  
\theoremstyle{definition_wo_parentheses}	
\newtheorem{definicion}{Definición}[subsection]

\theoremstyle{plain}	
\newtheorem{regla}{Regla}[subsection]

\theoremstyle{remark}	
\newtheorem{ejemplo}{Ejemplo}[subsection]


\begin{document}

% Título y autor
\maketitle

% Índice
\tableofcontents

\section{Resumen}

	El objetivo de esta práctica es la construcción de un sistema experto que ayude a un corredor de bolsa en la toma de decisiones diaria. El sistema tratará de maximizar el valor total de una cartera mediante operaciones de venta y compra de acciones. Para ello presentará al usuario las mejores propuestas basadas en el rendimiento que se espera obtener. 
	
	
\section{Descripción del proceso de desarrollo}

\subsection{Sesión 1}

	La primera sesión ha consistido en una entrevista con el directivo, el experto y el usuario sobre el sistema en general.
	
\paragraph{Entrevista con el directivo} Se establece que el problema consiste en ayudar a agentes novatos en su día a día para así facilitar el proceso de aprendizaje de éstos. El objetivo del sistema experto será mover acciones de una cartera para obtener la máxima rentabilidad. Se pretende también explicar qué hace y por qué, es decir, debe ser explicativo para que sea valioso.

\paragraph{Entrevista con el experto} Los datos de entrada serán cuestiones como el tipo del fondo que se esté manejando, la situación económica y política actual, la tendencia de la bolsa, las hipótesis que se realicen o los puntos de inflexión. Los datos disponibles serán la evolución de los valores en los últimos seis meses y en el último año y los datos de la bolsa de Madrid. El experto nos explica que existen tres tipos de clientes según el riesgo que están dispuestos a asumir. El experto analiza situaciones y valores para hacer un análisis previo de cada empresa y facilitar así la labor de decidir qué movimiento se realizará. Para ello estimará características como ser un valor estable o inestable. Se podrá utilizar la gráfica de la web de la bolsa para aplicar un filtro sobre valores con un rápido cambio, que serán aquellos que nos permitirán ganar o dejar de perder dinero en un poco espacio de tiempo. Nos indica que la bolsa está clasificada en sectores, que influyen en la toma de decisiones. Con el objetivo de diversificar la cartera, se establece que un valor no puede tener más del 50$\%$ de una cartera. El funcionamiento del sistema influye en el propio sistema pues debe corregir sus propios errores y permitir aumentar el riesgo si está obteniendo beneficios. El experto nos cuenta que para la toma de decisiones debemos tener en cuenta que existe un coste de un $0.5\%$ sobre la cantidad de dinero por cada operación. Los valores de riesgo donde haya invertido deberán ser los primeros en ser vendidos. 

\paragraph{Entrevista con el usuario} El usuario espera obtener una serie de cinco propuestas para realizar movimientos en la bolsa con el beneficio esperado y el riesgo de pérdida. Estos incluirán una explicación que permitan al usuario conocer la razón de la realización de la propuesta y así favorecer su aprendizaje. El usuario afirma que está dispuesto a introducir en el sistema la situación política del momento.\\


De esta reunión sacamos que la entrada consiste en:
\begin{itemize}
\item Cartera
\item Historial de los valores de bolsa
\item Situación
\end{itemize}

Las acciones concretas que realizará el programa serán:
\begin{itemize}
\item Deshacerse de los valores tóxicos
\item Descubrir valores prometedores
\item Clasificar los valores estables e inestables.
\item Clasificar las subidas y bajadas en duraderas o cortas
\end{itemize}


\subsection{Sesión 2}

	Las preguntas generales que tenemos para esta sesión con el experto son dos, \textit{¿cómo funciona la bolsa?} y \textit{¿en base a qué se decide invertir o no en un valor?} La respuesta del experto es que se trata de vender caro y comprar barato, aprovechar las subidas y bajadas.\\
	
	Los módulos en los que se dividirá el programa serán los siguientes:
	
	\begin{itemize}
		\item Entrada y petición de datos
		\item Salida y muestra de resultados
		\item Toma de decisiones para la realización de propuestas
		\item Previsión de tendencias
		\item Detección de grandes cambios. 
	\end{itemize}

	El proceso se llevará a cabo en varias fases:
	
	\begin{enumerate}
		\item Detección de valores peligrosos
		\item Detección de valores chollo
		\item Detección de valores sobrevalorados
		\item Fase de decisión
	\end{enumerate}
	
	Esto no lleva a la necesidad de incluir una primera definición de los términos que estamos usando:
	
	\begin{definicion}[Valores peligrosos] 
		Los valores peligrosos son aquellos que se prevee que caigan rápidamente. Estará relacionado con las noticias que se vayan produciendo.
	\end{definicion}

	\begin{definicion}[Valores sobrevalorados] 
		Los valores sobrevalorados son aquellos que se espera que vayan a converger a su valor real más bajo, más bajo. Es una estimación de la empresa a más largo plazo. 
	\end{definicion}
	
	Para la situación política se tienen en cuenta noticias generales sobre sectores o sobre un valor. Estas noticias afectarán positiva o negativamente. La actualización del SE se realizará a demanda por parte del usuario o tras un cambio sustancial. Para disponer de toda la información necesaria en el momento correspondiente, se realizará primero la detección de valores peligrosos y valores chollo, que se realizará continuamente. El análisis sobre la tendencia se realizará únicamente cuando se vaya a tomar una decisión. La detección de valores infravalorados y sobrevalorados se realizará en paralelo. En los módulos de detección de grandes cambios y la previsión de la tendencia usaremos el histórico de datos. En el módulo de decisiones se creará una lista de valores que merecen ser comprados y otra lista con aquellos a vender.\\
	
	El generador de propuestas concretas se actualizará con cada entrada. Este último módulo se basa el criterio de beneficio esperado. La estrategia de comprar barato y vender caro significa que hay una opción que nos beneficia. Es decir, esto puede ser quizá una única decisión que condense varios movimientos.
		
\subsection{Sesión 3}

	Se obtienen cuestiones generales del sistema y del módulo de detección de valores peligrosos.	
	
\begin{enumerate}
\item El sistema estará organizado en cuatro módulos:
	\begin{enumerate}[I]
	\item Toma de decisiones
	\item Detección de valores infravalorados
	\item Detección de valores chollo
	\item Detección de valores peligrosos
\end{enumerate}
\item Un valor se considerará peligroso si está cayendo más que la media de los valores de su sector.
\item Los sectores tienen distinto grado de estabilidad.
\item Dispondremos de la tendencia del sector.
\item El sistema tiene un módulo de arranque.
\item El sistema almacenará que ha habido una noticia en una fecha determinada.
\item Las noticias tienen distinto alcance. 
\item Las noticias con un determinado alcance influyen sobre los valores de ese ámbito. Los ámbitos de menor dimensión son más determinantes que los ámbitos superiores.
\item El módulo de detección de infravalorados y sobrevalorados no se ve afectado por las noticias y la inestabilidad.
\item Sólo observaremos los valores día a día, por lo que no influirán en la toma de decisiones los repuntes diarios.
\end{enumerate}
	
\begin{definicion}[Variación]
	La variación (caída/subida) es la diferencia entre el valor frente al que se está midiendo la variación y el valor actual.
\end{definicion} 

	Pasamos a tratar ahora sobre cuestiones específicas del módulo:
	
\begin{enumerate}
\item La salida del módulo es añadir hechos.
\item Las reglas serán simples.
\item Las entradas son los valores de las inversiones en los últimos días y su relación con los movimientos en cada sector.
\end{enumerate}

\begin{definicion}[Valor peligroso]
	Diremos que un valor es peligroso si cae más que el sector por cinco días.
\end{definicion}

\begin{ejemplo}
	Supongamos que durante cinco días el sector bancario baja un $5\%$ y Bankia baja un $10\%$. Entonces consideraremos que el valor de Bankia es peligroso.
\end{ejemplo}

\begin{regla}[Valor peligroso]
	Si un valor cae durante cinco días y ha caído más de un $5\%$ con respecto a su sector, se considera un valor peligroso.
\end{regla}

\paragraph{Estabilidad} La estabilidad de un determinado valor viene dada por las noticias sobre el valor y el sector al que pertenezca. La estabilidad influye, como veremos a continuación, en la toma de decisiones según sus valores. Un valor inestable que se vea afectado por una noticia positiva se considerará estable durante dos días a partir de la llegada de la noticia. Un valor que se considere estable pasará también a ser inestable por dos días si se introduce una noticia negativa que le afecte. Se tienen las siguientes consideraciones sobre los sectores:

\begin{enumerate}[a]
\item Servicios: Es inestable si la economía va mal.
\item Construcción: Sector inestable.
\end{enumerate}

\begin{regla}[Valor inestable peligroso]
	Si un valor inestable cae durante tres días y ha caído más de un $5\%$ con respecto a su sector, se considera un valor peligroso.
\end{regla}

\subsection{Diferentes sesiones prácticas}

	Durante el resto de sesiones se han ido planteando dudas al experto a partir de la documentación con el conocimiento del SE inversor conforme éstas iban surgiendo.
	
\paragraph{Estructura en módulos} La estructura en módulos del sistema es distinta a la que se extrajo del experto en la primera sesión. No hay módulo de detección de valores chollos y la 

\begin{enumerate}
\item[0] módulo de entrada de datos y deducción de valores inestables
\item[1] módulo de detección de valores peligrosos
\item[2] módulo de detección de valores infravalorados y sobrevalorados
\item[3] módulo de realización de propuestas
	\begin{enumerate}[a]
		\item obtención de posibles propuestas
		\item realización de posibles propuestas
	\end{enumerate}
\end{enumerate}
	
\begin{regla}[Valor inestable - Servicios]
	Si la economía está bajando, los valores del sector servicios son inestables por defecto. Se considerará, según el experto, que la economía está bajando si tiene pérdidas durante cinco días. 
\end{regla}	

\begin{regla}[Detección de valores infravalorados - Repunte]
	Si la empresa ha caído bastante (más de un $30\%$) en los últimos 3,6, ó 12 meses y ha subido pero no mucho (según el experto, menos de un $10\%$) en el último mes y el PER es bajo, la empresa está infravalorada.
\end{regla}	

\begin{regla}[Detección de valores infravalorados - Valor mejor que su sector]
	Si la empresa es grande, el RPD es alto y el PER Mediano, además no está bajando (ha crecido en el global de la última semana) y se comporta mejor que su sector (su variación en la última semana es mayor que la del sector), la empresa está infravalorada.
\end{regla}	

\begin{definicion}[Rendimiento esperado]
	Entendemos el rendimiento esperado como la suma de la revalorización anual esperada más los dividendos anuales esperados. Nos permitirá ordenar las posibles operaciones para su recomendación. 
\end{definicion}

\paragraph{Rendimiento esperado} Podemos estimar la revalorización anual esperada como el mejor valor de la variación en el último año, los últimos 6 ó 3 meses. El experto ha indicado que, debido a que en el último año la bolsa ha bajado, se propondría cambiar incluso buenos valores que tengamos, por otros cuyo RPD sea mayor que la suma del RPD del valor que disponemos más la variación anterior (negativa debido a la bajada de la bolsa). El experto, al observar que la propuesta de intercambiar es aún muy extensa, propone incluir la variación del último mes en la estimación de la revalorización anual esperada. Para las reglas dadas por el experto, calcularemos de forma distinta el rendimiento esperado según sus indicaciones, que repercutirá en el orden de selección de las medidas a proponer al usuario.

\begin{regla}[Inversión en empresas infravaloradas]
	 Si una empresa está infravalorada y el usuario tiene  dinero para invertir proponer invertir el dinero en las acciones de la empresa. Posteriormente se preguntará al usuario la cantidad a invertir y no se contemplan más reglas para decidir cuánto invertir en cada valor o sector.
\end{regla}	
	
\begin{definicion}[Precio del dinero]
	Tipo de interés fijado por el Banco Central Europeo para el préstamo de dinero a las entidades financieras. 
\end{definicion}

\begin{regla}[Venta de empresas sobrevaloradas]
	Si una empresa de mi cartera está sobrevalorada y el rendimiento por año $<5 +\mathbf{precio del dinero}$ proponer vender las acciones de esa empresa. Entendemos como el rendimiento por año la suma del RPD$\% +$ la variación del precio de las acciones que esperamos obtener en el año siguiente, que se puede estimar como la variación del último año.
\end{regla}	

\begin{regla}[Cambio por valores más rentables]
	 Si una empresa no está sobrevalorada y su RPD es mayor que el rendimiento esperado de una empresa de mi cartera que no está infravalorada más el $1\%$ de comisión por la transacción, proponer cambiar las acciones de una empresa por las de la otra. Aquí se calcula el rendimiento esperado de una empresa con la descripción anterior.
\end{regla}

\subsection{Procedimiento de validación y verificación del sistema seguido}

	El procedimiento de verificación ha consistido, en primer lugar, en probar la consistencia del sistema. Para ello se ha comprobado con cada regla que se lanza, descartando así que tengan consistencia estructural no disparable. Al correr el programa descartamos que haya un ciclo de reglas.  Así descartamos que haya inconsistencia estructural, pues la duplicación de reglas tampoco se da al haber condiciones distintas. Tampoco existe inconsistencia lógica, ya que como se explica en este informe el módulo 0 es el único lugar donde hay reglas que manejan los mismos hechos y conclusiones (noticias, valores inestables...) y se detalla porqué las reglas son necesaria y el funcionamiento es el correcto. Al comprobar cada regla se ha observado que el comportamiento es el correcto y los hechos que producen las reglas tienen sentido.\\
	
	El sistema también cumple con el proceso de validación pues tenemos una interfaz muy sencilla y que explica qué hay que introducir en cada situación para realizar las operaciones bursátiles. Cada hecho relacionado con las propuestas finalmente realizadas incluye una explicación de la regla que lo ha producido, por lo que el sistema es suficientemente explicativo. La ejecución es instantánea (ralentizada por la actualización de la base de hechos) así que la ejecución es en tiempo real.
	

\section{Descripción del sistema desarrollado}

\subsection{Variables de entrada del problema}

	\paragraph{Datos del IBEX35} El fichero \texttt{Analisis.txt} contiene la información del día anterior del IBEX35. Para cada valor de la bolsa, incluye una serie de atributos que usaremos en la deducción de nuevos datos y en la realización de propuestas.
	
	\begin{itemize}
		\item \textbf{Nombre} del valor
		\item \textbf{Precio} de la acción al cierre de la sesión
		\item \textbf{Variación} del precio de la acción con respecto al día anterior.
		\item \textbf{Capitalización}: Valor total de la empresa
		\item \textbf{PER}: Capitalización por beneficios anuales
		\item \textbf{RPD}: Reparto de dividendos
		\item \textbf{Tamaño} de la empresa
		\item \textbf{Etiqueta PER}: Categorización del PER
		\item \textbf{Etiqueta RPD}: Categorización del RPD
		\item \textbf{Sector}
		\item $\mathbf{\%}$\textbf{Variación 5 días}
		\item \textbf{Pérdidas en 3 días consecutivos}
		\item \textbf{Pérdidas en 5 días consecutivos}
		\item \textbf{Variación resp. al sector} durante 5 días.
		\item \textbf{Variación resp. al sector $<-5\%$}
		\item \textbf{Var. mensual}
		\item \textbf{Var. trimestral}
		\item \textbf{Var. semestral}
		\item \textbf{Var. anual}
	\end{itemize}
	
	Se incluirá en cada valor de la bolsa el rendimiento anual esperado, siguiendo la regla descrita anteriormente.
	
	
	\paragraph{Datos de los sectores} En el fichero \texttt{AnalisisSectores.txt} se encuentra la información relativa al global del IBEX35 y la media de los valores del sector.
	
	\begin{itemize}
		\item \textbf{Nombre} del sector
		\item \textbf{Variación} media de las empresas del sector.
		\item Suma de la \textbf{capitalización} de las empresas del sector.
		\item \textbf{PER} medio de las empresas del sector
		\item \textbf{RPD} medio de las empresas del sector
		\item $\mathbf{\%}$ \textbf{IBEX}
		\item $\mathbf{\%}$\textbf{Variación 5 días} media de las empresas del sector.
		\item \textbf{Pérdidas en 3 días consecutivos}
		\item \textbf{Pérdidas en 5 días consecutivos}
		\item \textbf{Var. mensual}
		\item \textbf{Var. trimestral}
		\item \textbf{Var. semestral}
		\item \textbf{Var. anual}
	\end{itemize}
	
\paragraph{Noticia} Las noticias se producen sobre valores concretos de la bolsa, sectores o sobre la economía en general. Afectan a la estabilidad de los valores durante dos días.

	\begin{itemize}
		\item \textbf{Nombre} del valor o sector al que le afecta. Si el nombre es ``Economía'' afectará a todos los valores.
		\item \textbf{Tipo} Buena/Mala
		\item \textbf{Antigüedad}: Días desde que se produjo la noticia.
	\end{itemize}

\paragraph{Cartera} La cartera representará la información que disponemos de las acciones y el dinero que tenemos en la cartera sin invertir. 

	\begin{itemize}
		\item \textbf{Nombre} del valor. El nombre para el dinero aún no invertido es \texttt{DISPONIBLE}
		\item \textbf{Acciones}: Número de acciones
		\item \textbf{Valor}: Número de acciones por el precio de cada acción
		\item \textbf{Actualizado}: Variable lógica que le añadimos a los datos de entrada para indicar si el valor de la cartera está actualizado. Esto facilitará llevar un seguimiento de la cartera con las modificaciones realizadas al hacer propuestas.
	\end{itemize}

	Además, declaramos en el hecho \texttt{(PrecioDinero 0)}, como constante, que podremos modificar, cuando así lo disponga el BCE en el fichero \texttt{Defini}-\texttt{cion.clp}. También se puede modificar en este fichero el número máximo de propuestas a realizar, actualmente declarado como 5.\\
	El resto de la entrada del programa se corresponde con la interacción del usuario.
	
\subsection{Variables de salida del problema}
	
	Las variables de salida son, fundamentalmente, las propuestas que se realizan al usuario. Estas incluyen la explicación que daría el experto sobre por qué las ha presentado. Cada hecho deducido por el sistema sobre los valores en bolsa o la economía en general incluye al hecho la explicación, lo que facilita la composición de la explicación de la propuesta final. Interiormente se deducen otros valores, como son los valores inestables, valores peligrosos, sobrevalorados o infravalorados, cuya definición,  incluida en la descripción de las sesiones, ayuda a comprender los motivos por los cuales se realizan las propuestas.\\
	En el módulo de interacción con el usuario, cuando éste va a salir del programa, tiene la posibilidad de guardar o no el fichero con la cartera de acciones que tiene en ese momento el sistema, lo cual constituye también una salida del programa. Para evaluar el sistema experto podríamos tomar la cartera final del día anterior y ver si vamos ganando dinero o no.
	
\subsection{Conocimiento global del sistema}

	El conocimiento global del sistema está compuesto por los datos de entrada y las reglas deducidas de la información extraída al experto. Se declaran también inicialmente las variables \texttt{(Contador (Indice 0))} y \texttt{(Suma (Suma 0))} para almacenar la información sobre el número de propuestas realizadas y el valor de la cartera que se irá acumulando. En el hecho \texttt{(ValorTotal (Valor))} almacenaremos el valor total de la cartera. Es también conocimiento global del sistema la localización de los ficheros de entrada.
	
\subsection{Estructura de funcionamiento del esquema de razonamiento}

\subsubsection{Hechos, reglas y funciones iniciales}
	
	Para facilitar la estructuración del programa en módulos, tanto a la hora de programar como para entender el funcionamiento, he dispuesto los módulos y funciones en diferentes ficheros, que se cargan de una vez con la función \texttt{Cargar}, del fichero \texttt{Cargar.clp}. En el fichero \texttt{Definicion.clp} se incluye la definición de los \texttt{template} descritos anteriormente, las variables y la función \texttt{dive} para realizar la división entera también con el tipo \texttt{double} que nos permita realizar de manera correcta los cálculos a la hora de cambiar valores de la cartera, realizar inversiones...\\
	Es en el fichero \texttt{Lectura.clp} donde entra en juego el esquema de razonamiento. Se define el hecho inicial con el nombre del fichero donde leer el análisis que disparará la ejecución. Se abre este fichero y se lee mientras no lleguemos al final del fichero, introduciendo en cada barrido el valor de la bolsa leído. Una vez que se finaliza, se cierra el fichero y se inserta el hecho que hará que se lea el análisis de los sectores. Se repite este proceso para leer el análisis de los sectores, las noticias y la cartera. Una vez se ha finalizado de leer la cartera, se entra en el módulo 0.\\
	
	El objetivo de esta parte del programa es únicamente establecer los valores iniciales y cargar en el programa los datos de entrada. No se ha deducido conocimiento más allá del incluido en los datos de entrada y que, una vez realizada las tareas preliminares, se está en el módulo 0.

	
\subsubsection{Módulo 0}
	
	En el módulo 0 se deduce conocimiento basándonos únicamente en las variables de entrada. En concreto, deducimos el valor RPA (Rendimiento esperado por año) siguiendo la regla dada por el experto. También deducimos los valores estables o inestables según las noticias y las peculiaridades de los valores del sector de la construcción y el sector servicios. La estrategia seguida ha sido marcar como inestable los valores cuando hay una noticia mala sobre su valor. No se ha introducido ningún tipo de prioridad para las reglas, por lo que hay que pensar si el funcionamiento es el esperado. Lo hacemos en orden descendente según la prevalencia de las reglas:
	
	\begin{description}
	\item[Noticias sobre valores] Las noticias malas sobre valores conllevan la declaración del valor como inestable. Las noticias buenas, eliminan el hecho de que un valor sea inestable. Por tanto, el hecho de que haya una noticia positiva sobre un valor tiene prioridad sobre cualquier regla que arroje el hecho de que un valor es inestable, y lo elimina.
	
	\item[Noticias sobre sectores] Las noticias malas sobre sectores conllevan la declaración de los valores de ese sector como inestables. Si esto se diese pero hubiera una noticia positiva sobre el valor concreto, se eliminaría este hecho, como se ha descrito anterioremente. Si hay una noticia positiva sobre un sector, tendríamos que eliminar el hecho de que cada valor del sector es inestable, salvo en el caso de que hubiera una noticia negativa sobre dicho valor en concreto, pues esto debería prevalecer. La otra forma de realizarlo sería declarando una prioridad mayor a las capas que menos influyen, primero a los valores por defecto, después a las noticias sobre la economía, las noticias sobre los sectores y por último las noticias sobre los valores para que permanezcan; sin embargo, de esta manera la estabilidad se decide por las reglas realizadas y no por el orden en el que se ejecutan.
	
	\item[Noticias sobre la economía] Por el conocimiento transmitido por el experto, la economía sólo puede afectar negativamente a los valores de la bolsa, luego consideramos únicamente esta situación en la que para cada valor de la bolsa, lo declaramos inestable. Las noticias positivas sobre sectores o valores descartarían este hecho.
	
	\item[Valores por defecto] De igual manera que las noticias sobre la economía en general, el valor por defecto de inestable para un valor o sector se ve modificado si hay una noticia positiva que afecte al sector o al valor.	
	\end{description}
	
	Una vez se hayan deducido los valores inestables, el contador con el número del módulo activo se actualiza a 1.
	
\subsubsection{Módulo 1}

	El objetivo de este módulo es la detección de los valores peligrosos. Necesita el conocimiento resultante del módulo 0 y el conocimiento que aporta al sistema son los hechos que indican qué valores son peligrosos. Se añade también una explicación sobre qué regla se ha activado para informar posteriormente al usuario del motivo de la propuesta realizada.\\
	
	Como todos los módulos, al finalizar el módulo actualiza el contador con el módulo activo y pasa al módulo 2.
	
	
\subsubsection{Módulo 2}

	El módulo 2 se encarga de la detección de valores sobrevalorados e infravalorados. Utiliza los hechos deducidos en los módulos anteriores y los datos de entrada de los valores. El conocimiento que aporta al sistema consiste en indicar qué valores están infravalorados y sobrevalorados, incluyendo la causa para mostrarle al usuario la explicación correspondiente.
	
\subsubsection{Módulo 3}

	Este módulo es el de mayor importancia en el programa, ya que se encarga de la realización de propuestas basándonos en la información proporcionada por el experto y por tanto tendrá una mayor influencia en las operaciones que realizará el usuario. \\
	El conocimiento que utiliza es el de los módulos anteriores. El conocimiento añadido al sistema consiste en la operación propuesta, el rendimiento esperado para dicha operación, las empresas involucradas en la operación y la explicación que se le mostrará al usuario. 
	
\subsubsection{Módulo 4}

	El módulo 4 tiene como objetivo la comunicación con el usuario. La funcionalidad de este módulo consiste en mostrar la mejor propuesta y decidir si se aplica o no, recalcular las propuestas a partir del estado de la cartera, actualizar los valores de las acciones que tenemos en la cartera según los datos de la bolsa, mostrando además el valor total de la cartera, y detener el programa, pudiendo guardar la cartera actual. Para ello usaremos los ejercicios de \texttt{CLIPS} que se pidieron con el menú, selección del máximo valor, eliminación del máximo, incremento de un contador...\\
	El conocimiento que requiere este módulo es todo el descrito anteriormente, además de la entrada introducida por el usuario. El conocimiento que aporta es la actualización de la cartera tanto por recalcular el valor total de las acciones según el precio del análisis de la bolsa (así podríamos ver la evolución de la cartera que sigue al realizar las operaciones propuestas sin descargar de nuevo el fichero \texttt{Cartera.txt}) como por los cambios introducidos por el usuario.
	
	
\subsection{Lista de hechos usados y representación}

\subsubsection{Hechos iniciales}

	Los hechos iniciales ya han sido descritos, se declara la constante del precio del dinero, el número máximo de propuestas, el contador con el número de propuestas realizadas y el hecho que iremos actualizando con el valor total de la cartera.
	
\begin{clips-code}
; ---------------------------------
; Template para llevar la suma del valor de las acciones
; ---------------------------------
(deftemplate Suma
  (field Suma)
)

; ---------------------------------
; Template para definir el valor total de la cartera
; ---------------------------------
(deftemplate ValorTotal
  (field Valor)
)

; Declaracion de las variables globales
(deffacts DeclaracionVariables
  (Contador (Indice 0))
  (Suma (Suma 0))
  (ValorTotal (Valor 0))
)

; Declaracion de la constante Precio del dinero
;   segun el precio establecido por el BCE
(deffacts PrecioDinero
  (PrecioDinero 0)
)

; Declaracion del numero de propuestas a realizar
(deffacts NumMaxPropuestas
  (NumMaxPropuestas 5)
)
\end{clips-code}

Comenzaremos la lectura de los ficheros de entrada por el análisis: 

\begin{clips-code}
; Declaracion del nombre del fichero de Analisis para su lectura
(deffacts readAnalisis
  (ReadAnalisis "Datos/Analisis.txt")
)
\end{clips-code}

El hecho que nos indicará que debemos seguir leyendo el archivo abierto es \texttt{(SeguirLeyendo)}. Para los demás archivos a leer, iremos afirmando o quitando los hechos

\begin{clips-code}
(ReadSectores "Datos/AnalisisSectores.txt")
(ReadNoticias "Datos/Noticias.txt")
(ReadCartera "Datos/Cartera.txt")
\end{clips-code}

Una vez que hayamos leído la cartera, se entra en el módulo 0 introduciendo el hecho \texttt{(Modulo (Indice 0))}.

\subsubsection{Módulo 0}

En el módulo 0 indicamos la estabilidad de los valores con hechos del tipo \texttt{(Inestable Nombre-valor Explicación)}. El resto  de hechos que usaremos será la existencia de noticias (buenas o malas), los valores de la bolsa y que la antigüedad sea menor a dos días. 

\subsubsection{Módulo 1}

En el módulo 1 se indican los valores peligrosos con hechos del tipo \texttt{(Peligroso Nombre-valor Explicación)}. Las reglas incluirán como condición el hecho \texttt{(Modulo (Indice 1))}.

\subsubsection{Módulo 2}

En el módulo 2 la estructura de los hechos utilizados es la misma que en los módulos previos, se indica si un valor está sobrevalorado o infravalorado con \texttt{(Infravalorado/Sobrevalorado Nombre-valor Explicación)}

\subsubsection{Módulo 3}

Los hechos relevantes en el módulo 3 son las propuestas. Su estructura es la siguiente

\begin{clips-code}

; ---------------------------------
; Template para una propuesta que realizaremos al usuario
; ---------------------------------
(deftemplate Propuesta
  ; Tipo de operacion
  (field Operacion)
  ; Empresa implicada
  (field Empresa)
  ; Rendimiento esperado
  (field RE)
  ; Motivo por el que se realiza la propuesta
  (field Explicacion)
  ; Segunda empresa implicada
  (field Empresa2
    (default NA))
  ; Valor logico que indica si la propuesta ya ha sido presentada
  (field Presentada
    (default false))
)
\end{clips-code}

	Las reglas se disparan únicamente si está activo el hecho \texttt{(Modulo (Indice 3))}
	
\subsubsection{Módulo 4}

En este módulo necesitamos más hechos para su funcionamiento. Comenzamos con las opciones para el menú, método que permite añadir fácilmente más funcionalidad independiente a la que ya tenemos. 

\begin{clips-code}
;----------------------------------------------------------------
; Opciones generales del menu
; ---------------------------------------------------------------
(deffacts OpcionesMenu
  (Menu 1 "Mostrar propuesta y valorar")
  (Menu 2 "Recalcular propuestas")
  (Menu 3 "Actualizar valor de la cartera")
  (Menu 0 "Detener programa")
  )
\end{clips-code}

El hecho que introducimos al entrar en el módulo 4 es \texttt{(PrintMenu)}, que hará que se muestren las opciones y el usuario decida una de ellas. Si la opción introducida es una de las mencionadas, se incluirá el hecho \texttt{(Respuesta X)}, que hará que se dispare la regla correspondiente a esta opción. Si no es una de las mencionadas, se volverá a introducid \texttt{(PrintMenu)}. Si se indica que queremos salir y guardar el fichero con el estado de la cartera, se introduce el hecho \texttt{(Guardar)}, que hará que vayamos escribiendo en el fichero \texttt{Datos/CarteraMod.txt} la cartera (podríamos cambiar en la regla \texttt{Salir} el fichero donde guardarlo, está así para no sobreescribir el fichero \texttt{Cartera.txt}). Una vez se ha guardado la cartera o se ha indicado que no se pretende guardar, no se afirma nada y por tanto se detiene la ejecución. \\

Para mostrar el mejor resultado (con el hecho \texttt{(Respuesta 1)}), comprobamos que el contador lleve menos de \texttt{(NumMaxPropuestas 5)} propuestas y seleccionamos aquella no presentada con mayor rendimiento. Se le pregunta al usuario si desea llevar a cabo la operación propuesta. Si la respuesta es afirmativa, se introduce un hecho \texttt{(Operacion Tipo-Operación Empresa1 Empresa2)}. Cuando se introducen estos hechos debemos actualizar la cartera y las propuestas realizadas. Para ello, quitamos los hechos de la cartera si vendemos acciones y añadimos si las compramos. En el primer caso, puesto que hemos comprado acciones (o realizado un intercambio), descartaremos las propuestas que se traten de comprar más acciones de esta empresa. Si lo que hemos hecho ha sido vender acciones o intercambiar acciones de esta empresa que tenemos en la cartera por otra más rentable, descartamos aquellas propuestas de venta de estas acciones que ya no disponemos. El hecho usado será \texttt{(Descartar Compra/Venta Nombre-valor)}\\

Para recalcular propuestas según el estado de la cartera, tenemos el hecho \texttt{(Respuesta 2)}, por el que en primer lugar eliminamos las propuestas existentes y posteriormente volvemos al módulo 1.\\

La última funcionalidad consiste en la actualización de la cartera. Se dispara con el hecho \texttt{(Respuesta 3)} y se modifica cada valor de la cartera, poniendo el campo \texttt{(Atributo)} a \texttt{false}. Para, una vez que hayamos realmente actualizado el valor de las acciones que tenemos no se vuelva a disparar la regla, eliminamos el hecho de la respuesta e introducimos el hecho \texttt{(ActualizandoCartera)}, que dispara la regla que actualiza cada valor y va acumulando el valor total de la cartera. Una vez que no quedan valores por actualizar, se elimina el hecho \texttt{(ActualizandoCartera)} y muestra la suma total.\\

Cada vez que realizamos una acción que no sea la salida del programa, se muestra el menú del programa para que seleccionemos la acción a ejecutar.

\subsection{Reglas de cada módulo}

	Como ya los hechos han sido ampliamente descritos, se describen las reglas concretas de cada módulo.
	
\subsubsection{Lectura}

\paragraph{openAnalisis} Es la primera regla que se dispara, para lo que tiene una prioridad de 50. Con el hecho que indica que hay que leer el análisis y el nombre del fichero, abre el fichero y afirma el hecho \texttt{(SeguirLeyendo)} descrito anteriormente.

\paragraph{readingAnalisis} Con una prioridad de 49 y mientras se indique que el fichero a leer es el del análisis y esté activo el hecho \texttt{(SeguirLeyendo)}, realizamos el proceso de leer. Si recibimos algo distinto del fin del fichero, procedemos a leer cada atributo del valor de la bolsa, lo introducimos en la lista de hechos y volvemos a introducir el hecho \texttt{(SeguirLeyendo)}. 

\paragraph{closeFile} Cuando hemos terminado de leer este fichero, con prioridad 48, cerramos el \textit{stream} \texttt{mydata}, eliminamos el hecho que indica que debemos leer el fichero de análisis e introducimos el hecho que indica que leeremos el fichero con el análisis de los sectores. 

Análogas a estas están las reglas \texttt{openSectores, readingSectores, closefileSectores,openNoticias, readingNoticias, closefileNoticias, openCartera, readingCartera} y  \texttt{closeFileCartera} el cuál da paso al módulo 0. La prioridad va bajando con respecto al orden en el que se deben ejecutar las reglas. 

\subsubsection{Módulo 0}

Todas las reglas de este módulo incluyen en el antecedente el hecho de que el módulo activo sea el 0.

\paragraph{RPA} Para simplificar las reglas de los demás módulos (principalmente del módulo 3), se calcula el rendimiento esperado por año. Para cada valor que tenga el RPA como NA, se suma el RPD$\%$ al máximo de la variación anual, semestral, trimestral y mensual y se modifica el valor con el RPA correcto.

\paragraph{DefectoConstrucción} Para cada valor de la construcción, se introduce el hecho de que el valor sea inestable si no hay una noticia que lo vuelva inestable, con la explicación de que pertenece al sector de la construcción.

\paragraph{DefectoServicios} Para cada valor del sector servicios, si el IBEX lleva 5 días consecutivos cayendo y no hay una noticia que lo vuelva inestable, se introduce el hecho de que el valor es inestable y la explicación de que pertenece al sector servicios y la economía está bajando.

\paragraph{InestableEconomía} Si hay una noticia negativa sobre la economía, para cada valor se introduce el hecho de que el valor es inestable si no hay una noticia mala más específica y el hecho de que hay una noticia mala sobre la economía como explicación.

\paragraph{InestableSector} Si hay una noticia mala sobre un sector, un valor es de ese sector y no hay una noticia mala concreta sobre ese valor, se introduce el hecho de que ese valor es inestable junto con la explicación que indica el sector del que ha habido tal noticia.

\paragraph{EstableSector} Si hay una noticia buena sobre un sector y hay un valor que es inestable y no hay también una noticia pero negativa sobre el valor en concreto, quitaremos el hecho de que el valor sea inestable. 

\paragraph{InestableValor} Si hay una noticia negativa sobre un valor, se afirmará el hecho de que el valor es inestable con la explicación correspondiente.

\paragraph{EstableValor} Si hay una noticia positiva sobre un valor marcado como inestable, se eliminará este hecho.

Para todos los casos, se comprueba que la antigüedad de la noticia sea menor a dos días. 

\paragraph{SalirMódulo0} Con prioridad -1, se sale del módulo 0 y se entra en el módulo 1.

\subsubsection{Módulo 1} 

\paragraph{DetecciónPeligrosoInestable} Para los valores inestables que lleven tres días consecutivos de pérdidas se afirma que ese valor es peligroso y por qué, explicando también por qué es inestable.

\paragraph{DetecciónPeligroso} Para los valores no inestables, se afirma que es peligroso si lleva 5 días consecutivos de pérdidas, incluyendo este motivo.

\paragraph{SalirMódulo1} Con prioridad -1, se sale del módulo 1 y se entra en el módulo 2.


\subsubsection{Módulo 2}

\paragraph{DetecciónSobrevaloradosGeneral} Para los valores con PER alto y RPD bajo se indica que está sobrevalorado y por qué. 

\paragraph{DetecciónSobrevaloradosPeq1} Para los valores pequeños con PER alto se indica que el valor está sobrevalorado por este motivo. Como estas dos reglas se pueden dar simultáneamente, incluimos que se lancen sólo si el valor no ha sido marcado ya como sobrevalorado, ya que el funcionamiento posterior es el mismo y sólo varía la explicación, siendo ciertos ambos motivos. 

\paragraph{DetecciónSobrevaloradosPeq2} Si un valor pequeño tiene PER mediano y RPD bajo, se indica que el valor está sobrevalorado por esta causa.

\paragraph{DetecciónSobrevaloradosGrande1} Si un valor grande tiene PER mediano y RPD bajo, indicamos que el valor está sobrevalorado.

\paragraph{DetecciónSobrevaloradosGrande2} Si un valor grande tiene RPD mediano y PER alto, introducimos el hecho de que esté sobrevalorado por este motivo.

\paragraph{DetecciónInfravalorados1} Si un valor tiene el PER bajo y el RPD alto, indicaremos que este valor está infravalorado por esta causa. 

\paragraph{DetecciónInfravalorados2} Si un valor tiene el PER bajo, ha bajado en el último trimestre, semestre o año más de un 30$\%$ y ha subido menos de un $10\%$ en el último mes, la empresa está infravalorada y así lo indicamos, añadiendo esta causa. Al igual que con los sobrevalorados, estas dos reglas son compatibles, así que lanzamos únicamente una.

\paragraph{DetecciónInfravalorados3} Si una empresa grande tiene un RPD alto, un PER mediano, ha crecido en los últimos 5 días y se comporta en estos cinco días mejor que su sector, añadimos un hecho indicando que la empresa está infravalorada.

\paragraph{SalirMódulo2} Con prioridad -1, se sale del módulo 2 y se entra en el módulo 3.

\subsubsection{Módulo3} 

\paragraph{VentaPeligrosos} Si una empresa de la cartera es considerada peligrosa, ha caído en el último mes y la diferencia con respecto a su sector es menor que un $-3\%$, realizamos la propuesta de vender las acciones de la empresa incluyendo la explicación dada y con el rendimiento esperado de $20 - RPD\%$.

\paragraph{InversiónInfravalorados} Si una empresa está infravalorada, tenemos dinero disponible en la cartera para comprar al menos una acción y su PER no es 0 (para poder calcular el rendimiento esperado), proponemos invertir en esta empresa. Comprobamos también que, en caso de que tengamos acciones de la empresa, el valor total de éstas sea menor que la mitad del valor total de la cartera. El RE es $ 20 \frac{PER_{medio} - PER}{PER} + RPD\%$.

\paragraph{VentaSobrevalorados} Si tenemos acciones de una empresa sobrevalorada cuyo RPA sea menor que un $5\% + $ \textit{Precio del dinero}, y PER distinto de 0, propondremos vender las acciones de la empresa. Su RE es  $ 20 \frac{PER - PER_{medio}}{PER} - RPD\%$. Es distinto al que dijo en un principio el experto, pero durante la validación del sistema se obtuvieron propuestas con rendimiento negativo, por lo que se le consultó al experto y éste indicó que a la fórmula le faltaba multiplicar la diferencia entre el $PER$ y el $PER_{medio}$ del sector por 100. 

\paragraph{CambiarInversión} Si tenemos una empresa en la cartera que no está infravalorada y hay una empresa no sobrevalorada cuyo RPD$\%$ es mayor que el RPA de nuestra empresa +1, propondremos cambiar la inversión. Comprobamos también previamente que, en caso de que ya  tengamos acciones de la empresa por la que vamos a intercambiar nuestras acciones, el valor total de éstas sea menor que la mitad del valor total de la cartera. Se le resta a lo invertido en la otra empresa el $1\%$ de comisión. El RE es $1+RPA_2-RPD_1\%$.

\paragraph{SalirMódulo3} Con prioridad -1, se sale del módulo 3 y se entra en el módulo 4

\subsubsection{Módulo 4}

\paragraph{Menú} Es la regla que se dispara cuando se entra en el módulo 4. Introduce el hecho para sacar el menú por pantalla. 

\paragraph{PrintOption} Con prioridad 1, cuando está la regla de imprimir el menú, para cada opción del menú sale por pantalla.

\paragraph{ReadAnswer} Cuando ya se han imprimido todas las opciones, se recoge la respuesta. 

\paragraph{NoOption} Si la respuesta recogida no está en el menú, se muestra un mensaje diciendo que la opción es incorrecta y se indica el hecho para sacar por pantalla el menú de nuevo.

\paragraph{Salir} Si la respuesta es 0 se pide confirmación para salir del programa, si la respuesta es \texttt{S} o \texttt{s}, pregunta si se quiere guardar el estado de la cartera. En caso de que la respuesta sea afirmativa, se abre el fichero \texttt{Datos/CarteraMod.txt} y se introduce el hecho \texttt{(Guardar)}, si no, simplemente se detiene la ejecución. Si la respuesta a la salida es negativa, se vuelve al menú. Si quisiéramos salir realmente del entorno \texttt{CLIPS} y no sólo detener la ejecución podríamos incluir una regla con prioridad -1 para haga una llamada a la función \texttt{(exit)}).

\paragraph{guardarDisponible} Con prioridad 3, si se ha indicado que se desea guardar la cartera, se toma el valor del dinero disponible y se escribe en el fichero (así nos aseguramos que el fichero con la cartera mantenga el formato).

\paragraph{guardarCartera} Con prioridad 2, si se ha indicado que se pretende guardar la cartera, para cada valor de la cartera distinto del dinero disponible, se escribe en el fichero.

\paragraph{cerrarCarteraSalir} Una vez que se han escrito todos los valores de la cartera, se cierra el fichero y se elimina el hecho \texttt{(Guardar)}.

\paragraph{MostrarMejorResultado} Si la respuesta ha sido la opción 1 se comprueba que el contador con el número de propuestas realizadas sea menor que el número máximo establecido, se toma la propuesta no presentada con mayor RE, es decir, aquella propuesta no presentada tal que no haya otra con un RE mayor. Entonces se elimina el hecho con la respuesta del menú; se pide confirmación para realizar la operación;  si la respuesta es positiva, se elimina la propuesta y se indica que debe realizarse la operación; si es negativa, se indica que la propuesta ya ha sido presentada. En cualquier caso, se aumenta en 1 el contador de propuestas realizadas y se vuelve al menú.

\paragraph{SinPropuestas} Si la respuesta en el menú es 1 pero no hay propuestas o el número de propuestas realizadas es mayor o igual al número máximo de propuestas, se muestra un mensaje por pantalla y se vuelve al menú. 

\paragraph{VentaAcciones} Si hay una operación de venta de acciones de una empresa que tenemos en cartera, actualizamos el dinero disponible en la cartera, quitamos el valor de la cartera e introducimos el hecho que nos permita descartar la venta de las acciones que ya hemos vendido.

\paragraph{EfectuarInversiónNuevoValor} Si hay una operación de inversión en una empresa de la no tenemos acciones, se le pide al usuario que introduzca el dinero que pretende invertir en la empresa. Se calcula el número de acciones que se puede comprar con ese dinero (teniendo en cuenta la comisión) y se añade a la cartera el hecho con el número de acciones compradas y su valor. Se resta del dinero disponible lo que finalmente se gasta en las acciones. Se descartan también las propuestas que significan la compra de más acciones de esta empresa para así tratar de diversificar la cartera.

\paragraph{EfectuarInversión} Si hay una operación de inversión en una empresa de la que ya tenemos acciones, se le pide al usuario que introduzca el dinero que pretende invertir en la empresa. Se calcula el número de acciones que se puede comprar con ese dinero (teniendo en cuenta la comisión) y se añade al hecho que tenemos en la cartera con las acciones de esta empresa las acciones que hemos comprado y su valor. Se resta del dinero disponible lo que finalmente se gasta en las acciones. Se descartan también las propuestas que significan la compra de más acciones de esta empresa para así tratar de diversificar la cartera.

\paragraph{IntercambioValoresNuevoValor} Si hay una operación de intercambio entre dos valores y no tenemos acciones del valor del que vamos a comprar acciones, se eliminan de la cartera las acciones que vendemos, calculamos el número de acciones a comprar teniendo en cuenta las dos comisiones que debemos pagar ahora. Sumamos al dinero disponible en la cartera si hubiera algún pico (sólo compramos un número entero de acciones). Descartamos la venta de las acciones que ya no poseemos y la compra de más acciones de la empresa de la que hemos adquirido. Añadimos el hecho a la cartera con el número de acciones y su valor.


\paragraph{IntercambioValores} Si hay una operación de intercambio entre dos valores y ya tenemos acciones del valor del que vamos a comprar acciones, se eliminan de la cartera las acciones que vendemos, calculamos el número de acciones a comprar teniendo en cuenta las dos comisiones que debemos pagar ahora. Sumamos al dinero disponible en la cartera si hubiera algún pico (sólo compramos un número entero de acciones). Descartamos la venta de las acciones que ya no poseemos y la compra de más acciones de la empresa de la que hemos adquirido. Modificamos el hecho de la cartera con el número total de acciones y el total del valor.

\paragraph{DescartarVentaTrasOperación} Para toda empresa que se haya indicado que debemos descartar su venta, eliminamos las propuestas de ventas.

\paragraph{DescartarIntercambioTrasOperación} Para toda empresa que se haya indicado que debemos descartar su venta, eliminamos las propuestas de intercambio de las acciones que poseíamos de dicha empresa.

\paragraph{StopDescartarVenta} Si no hay propuestas de ventas o intercambios de una empresa sobre la que hemos indicado que debemos descartar los hechos, se elimina el hecho de descartar ventas de la empresa. Posteriormente se accede al menú.

\paragraph{DescartarCompraTrasOperación} Si hemos descartado la compra de acciones de una empresa, eliminamos los hechos que signifiquen la inversión o el intercambio de acciones de esta empresa por una que poseamos.

\paragraph{DescartarCompraInsuficienciaDinero} Si hemos realizado una compra y hay propuestas de realizar una inversión en una empresa pero el precio de la acción (más el 0.5$\%$ de comisión) es mayor que el dinero disponible en la cartera, descartaremos esta propuesta.


\paragraph{StopDescartarCompra} Si hemos introducido el hecho de descartar la compra de una empresa y ya no hay propuestas de inversión o intercambio de acciones de esa empresa, se elimina el hecho de descartar estas operaciones sobre la empresa. Se accede al menú.

\paragraph{RecalcularEliminarPropuestas} Si la respuesta en el menú es 2, se elimina cada propuesta existente. 

\paragraph{Recalcular} Si no quedan propuestas por eliminar, se pone el contador de las propuestas realizadas a 0 y se pasa al módulo 1 para recalcular propuestas.

\paragraph{DisponerActualizaciónValorCartera} Con prioridad 4, si la respuesta en el menú es 3, para cada valor de la cartera que no esté actualizado se marca el campo \texttt{(Actualizado)} a \texttt{false}.

\paragraph{EvitarBucle} Con prioridad 3, se elimina el hecho de \texttt{(Respuesta 3)} y se introduce el hecho \texttt{(ActualizandoCartera)}. Así evitamos volver a actualizar un valor ya actualizado.

\paragraph{ActualizarValorCartera} Con prioridad 2, si estamos actualizando la cartera y hay un valor sin actualizar, se toma el precio de la acción del análisis y el número de acciones en la cartera, actualizando el valor en la cartera. Se suma al hecho \texttt{(Suma)} para mostrar el valor total de la cartera.

\paragraph{MostrarValorCartera} Con prioridad 1 (ya se han actualizado todos los valores) se muestra la suma del valor total de la cartera y se vuelve al menú.

\section{Manual de uso}

Como se ha comentado, se ha distribuido el código en varios ficheros, facilitando la localización de los hechos y reglas, pudiendo cambiarlos por separado. Para cargar todos los módulos del programa está el fichero \texttt{Cargar.clp}, que contiene la definición de la función \texttt{(Cargar)}, que carga todos los módulos (las definiciones de las reglas, \texttt{templates} y declaración de hechos), \textit{resetea} los hechos y corre el programa. Si queremos no cargar algún módulo o no hacer que la ejecución sea automática, podemos comentar la línea correspondiente o hacerlo manualmente, cargando en primer lugar el fichero \texttt{Definicion.clp} y después los módulos deseados.\\

Si ejecutamos la función \texttt{(Cargar)}, ya sólo tendremos que seguir s inslatrucciones del programa, indicando la opción a tomar deseada, teniendo en cuenta que para la confirmación en la toma de acciones, si queremos realizarla escribiremos \texttt{s} o \texttt{S}, y si no, cualquier otra tecla. Si estamos en el menú general y la opción introducida no se corresponde con ninguna de las existentes, el menú volverá a pedir una opción válida. Como se ha comentado, la entrada de la cartera es el fichero \texttt{Cartera.txt} y donde se guarda la cartera existente en ese momento cargada en el programa se guarda en el fichero \texttt{Datos/CarteraMod.txt}.

\end{document}